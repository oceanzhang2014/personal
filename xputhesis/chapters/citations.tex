% !TeX root = ../main.tex

\chapter{引用文献的标注}

模板使用 \pkg{natbib} 宏包来设置参考文献引用的格式,
更多引用方法可以参考该宏包的使用说明\cite{PhysRevA.72.063613}。



\section{顺序编码制}

\subsection{角标数字标注法}

\ustcsetup{
  cite-style = super,
}
\noindent
\begin{tabular}{l@{\quad$\Rightarrow$\quad}l}
  \verb|\cite{knuth86a}|         & \cite{knuth86a}         \\
  \verb|\citet{knuth86a}|        & \citet{knuth86a}        \\
  \verb|\cite[42]{knuth86a}|     & \cite[42]{knuth86a}     \\
  \verb|\cite{knuth86a,tlc2}|    & \cite{knuth86a,tlc2}    \\
  \verb|\cite{knuth86a,knuth84}| & \cite{knuth86a,knuth84} \\
\end{tabular}


\subsection{数字标注法}

\ustcsetup{
  cite-style = inline,
}
\noindent
\begin{tabular}{l@{\quad$\Rightarrow$\quad}l}
  \verb|\cite{knuth86a}|         & \cite{knuth86a}         \\
  \verb|\citet{knuth86a}|        & \citet{knuth86a}        \\
  \verb|\cite[42]{knuth86a}|     & \cite[42]{knuth86a}     \\
  \verb|\cite{knuth86a,tlc2}|    & \cite{knuth86a,tlc2}    \\
  \verb|\cite{knuth86a,knuth84}| & \cite{knuth86a,knuth84} \\
\end{tabular}



\section{著者-出版年制标注法}

\ustcsetup{
  cite-style = authoryear,
}
\noindent
\begin{tabular}{l@{\quad$\Rightarrow$\quad}l}
  \verb|\cite{knuth86a}|         & \cite{knuth86a}         \\
  \verb|\citep{knuth86a}|        & \citep{knuth86a}        \\
  \verb|\citet[42]{knuth86a}|    & \citet[42]{knuth86a}    \\
  \verb|\citep[42]{knuth86a}|    & \citep[42]{knuth86a}    \\
  \verb|\cite{knuth86a,tlc2}|    & \cite{knuth86a,tlc2}    \\
  \verb|\cite{knuth86a,knuth84}| & \cite{knuth86a,knuth84} \\
\end{tabular}

\ustcsetup{
  cite-style = super,
}

% 注意,参考文献列表中的每条文献在正文中都要被引用。这里只是为了示例。
\nocite{*}
