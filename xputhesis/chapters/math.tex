% !TeX root = ../main.tex

\chapter{数学}

\begin{eqnarray}
	\alpha & = & \varepsilon \sqrt{a} 
\end{eqnarray}

\begin{eqnarray}
	H=&&\hbar \omega_{c} c^{+} c+\frac{1}{2} \hbar \omega_{m}\left(X^{2}+P^{2}\right)+\hbar \sum_{i=1}^{N}\left(\omega_{10} \sigma_{11}^{i}+\omega_{20} \sigma_{22}^{i}\right) \\
	&&-\hbar g_{1} c^{+} c X+\hbar T_{e} \sum_{i=1}^{N}\left(\sigma_{12}^{i}+\sigma_{21}^{i}\right)+\hbar g \sum_{i=1}^{N}\left(c \sigma_{10}^{i}+c^{+} \sigma_{01}^{i}\right) \\
	&&+i \hbar \varepsilon_{l}\left(c^{+} e^{-i \omega_{l} t}-c e^{i \omega_{l} t}\right)+i \hbar \varepsilon_{p}\left(c^{+} e^{-i \omega_{p} t}-c e^{i \omega_{p} t}\right)
\end{eqnarray}

\section{数学符号}

《撰写手册》\cite{PhysRevLett.81.3811,baker1995future}要求数学符号遵循 GB/T 3102.11—1993《物理科学和技术中使用的数学符号》
\footnote{原 GB 3102.11—1993,自 2017 年 3 月 23 日起,该标准转为推荐性标准。}。
该标准参照采纳 ISO 31-11:1992 \footnote{目前已更新为 ISO 80000-2:2019。},
但是与 \TeX{} 默认的美国数学学会(AMS)的符号习惯有所区别。
具体地来说主要有以下差异:
\begin{enumerate}
  \item 大写希腊字母默认为斜体,如
    \begin{equation*}
      \Gamma \Delta \Theta \Lambda \Xi \Pi \Sigma \Upsilon \Phi \Psi \Omega.
    \end{equation*}
    注意有限增量符号 $\increment$ 固定使用正体,模板提供了 \cs{increment} 命令。
  \item 小于等于号和大于等于号使用倾斜的字形 $\le$、$\ge$。
  \item 积分号使用正体,比如 $\int$、$\oint$。
  \item
    偏微分符号 $\partial$ 使用正体。
  \item
    省略号 \cs{dots} 按照中文的习惯固定居中,比如
    \begin{equation*}
      1, 2, \dots, n \quad 1 + 2 + \dots + n.
    \end{equation*}
  \item
    实部 $\Re$ 和虚部 $\Im$ 的字体使用罗马体。
\end{enumerate}

以上数学符号样式的差异可以在模板中统一设置。
但是还有一些需要用户在写作时进行处理:
\begin{enumerate}
  \item 数学常数和特殊函数名用正体,如
    \begin{equation*}
      \uppi = 3.14\dots; \quad
      \symup{i}^2 = -1; \quad
      \symup{e} = \lim_{n \to \infty} \left( 1 + \frac{1}{n} \right)^n.
    \end{equation*}
  \item 微分号使用正体,比如 $\dif y / \dif x$。
  \item 向量、矩阵和张量用粗斜体(\cs{symbf}),如 $\symbf{x}$、$\symbf{\Sigma}$、$\symbfsf{T}$。
  \item 自然对数用 $\ln x$ 不用 $\log x$。
\end{enumerate}

模板中使用 \pkg{unicode-math} 宏包配置数学字体。
该宏包与传统的 \pkg{amsfonts}、\pkg{amssymb}、\pkg{bm}、
\pkg{mathrsfs}、\pkg{upgreek} 等宏包\emph{不}兼容。
本模板作了处理,用户可以直接使用 \cs{bm}, \cs{mathscr},
\cs{upGamma} 等命令。
关于数学符号更多的用法,参见 \pkg{unicode-math} 宏包的使用说明和符号列表
\pkg{unimath-symbols}。



\section{数学公式}

数学公式可以使用 \env{equation} 和 \env{equation*} 环境。
注意数学公式的引用应前后带括号,建议使用 \cs{eqref} 命令,比如式~\eqref{eq:example}。
\begin{equation}
  \hat{f}(\xi) = \int_{-\infty}^\infty f(x) \eu^{-2 \uppi \iu x \xi} \dif x.
  \label{eq:example}
\end{equation}

多行公式尽可能在“=”处对齐,推荐使用 \env{align} 环境,比如式~\eqref{eq:align_2}。
\begin{align}
  a & = b + c + d + e \label{eq:align_1} \\
    & = f + g. \label{eq:align_2}
\end{align}



\section{量和单位}

量和单位要求严格执行 GB 3100~3102—1993 有关量和单位的规定。
宏包 \pkg{siunitx} 提供了更好的数字和单位支持:
\begin{itemize}
  \item 为了阅读方便,四位以上的整数或小数推荐采用千分空的分节方式:\num{55235367.34623}。
    四位以内的整数可以不加千分空:\num{1256}。
  \item 数值与单位符号间留适当空隙:\SI{25.4}{mm},\SI{5.97e24}{\kilo\gram},
    \SI{-273.15}{\degreeCelsius}。 例外:\SI{12.3}{\degree},\ang{1;2;3}。
  \item 组合单位默认使用 APS 的格式,即相乘的单位之间留一定空隙: \si{kg.m.s^{-2}},
    也可以使用居中的圆点: \si[inter-unit-product = \ensuremath{{}\cdot{}}]{kg.m.s^{-2}}。
    GB 3100—1993 对两者都允许,建议全文统一设置。
  \item 量值范围使用“~”:\SIrange{10}{15}{mol/L}。
  \item 注意:词头 \textmu{} 不能写为 u,如:\si{umol} 应为 \si{\micro\mole}、\si{\umol}。
\end{itemize}



\section{定理和证明}

示例文件中使用 \pkg{amsthm} 宏包配置了定理、引理和证明等环境。
用户也可以使用 \pkg{ntheorem} 宏包。

\begin{definition}
  If the integral of function $f$ is measurable and non-negative, we define
  its (extended) \textbf{Lebesgue integral} by
  \begin{equation}
    \int f = \sup_g \int g,
  \end{equation}
  where the supremum is taken over all measurable functions $g$ such that
  $0 \le g \le f$, and where $g$ is bounded and supported on a set of
  finite measure.
\end{definition}

\begin{assumption}
The communication graph is strongly connected.
\end{assumption}

\begin{example}
  Simple examples of functions on $\mathbb{R}^d$ that are integrable
  (or non-integrable) are given by
  \begin{equation}
    f_a(x) =
    \begin{cases}
      |x|^{-a} & \text{if } |x| \le 1, \\
      0        & \text{if } x > 1.
    \end{cases}
  \end{equation}
  \begin{equation}
    F_a(x) = \frac{1}{1 + |x|^a}, \qquad \text{all } x \in \mathbb{R}^d.
  \end{equation}
  Then $f_a$ is integrable exactly when $a < d$, while $F_a$ is integrable
  exactly when $a > d$.
\end{example}

\begin{lemma}[Fatou]
  Suppose $\{f_n\}$ is a sequence of measurable functions with $f_n \geq 0$.
  If $\lim_{n \to \infty} f_n(x) = f(x)$ for a.e. $x$, then
  \begin{equation}
    \int f \le \liminf_{n \to \infty} \int f_n.
  \end{equation}
\end{lemma}

\begin{remark}
  We do not exclude the cases $\int f = \infty$,
  or $\liminf_{n \to \infty} f_n = \infty$.
\end{remark}

\begin{corollary}
  Suppose $f$ is a non-negative measurable function, and $\{f_n\}$ a sequence
  of non-negative measurable functions with
  $f_n(x) \le f(x)$ and $f_n(x) \to f(x)$ for almost every $x$. Then
  \begin{equation}
    \lim_{n \to \infty} \int f_n = \int f.
  \end{equation}
\end{corollary}

\begin{proposition}
  Suppose $f$ is integrable on $\mathbb{R}^d$. Then for every $\epsilon > 0$:
  \begin{enumerate}
    \renewcommand{\theenumi}{\roman{enumi}}
    \item There exists a set of finite measure $B$ (a ball, for example) such
      that
      \begin{equation}
        \int_{B^c} |f| < \epsilon.
      \end{equation}
    \item There is a $\delta > 0$ such that
      \begin{equation}
        \int_E |f| < \epsilon \qquad \text{whenever } m(E) < \delta.
      \end{equation}
  \end{enumerate}
\end{proposition}

\begin{theorem}
  Suppose $\{f_n\}$ is a sequence of measurable functions such that
  $f_n(x) \to f(x)$ a.e. $x$, as $n$ tends to infinity.
  If $|f_n(x)| \le g(x)$, where $g$ is integrable, then
  \begin{equation}
    \int |f_n - f| \to 0 \qquad \text{as } n \to \infty,
  \end{equation}
  and consequently
  \begin{equation}
    \int f_n \to \int f \qquad \text{as } n \to \infty.
  \end{equation}
\end{theorem}

\begin{proof}
  Trivial.
\end{proof}

\newtheorem*{axiomofchoice}{Axiom of choice}
\begin{axiomofchoice}
  Suppose $E$ is a set and ${E_\alpha}$ is a collection of
  non-empty subsets of $E$. Then there is a function $\alpha
  \mapsto x_\alpha$ (a ``choice function'') such that
  \begin{equation}
    x_\alpha \in E_\alpha,\qquad \text{for all }\alpha.
  \end{equation}
\end{axiomofchoice}

\newtheorem{observation}{Observation}
\begin{observation}
  Suppose a partially ordered set $P$ has the property
  that every chain has an upper bound in $P$. Then the
  set $P$ contains at least one maximal element.
\end{observation}
\begin{proof}[A concise proof]
  Obvious.
\end{proof}
